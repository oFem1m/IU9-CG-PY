\documentclass[a4paper, 14pt]{extarticle}

% Поля
%--------------------------------------
\usepackage{geometry}
\usepackage[T2A]{fontenc}
\usepackage[russian]{babel}
\usepackage{minted}
\usepackage{float}
\usepackage{graphicx} 
\geometry{a4paper,tmargin=2cm,bmargin=2cm,lmargin=3cm,rmargin=1cm}
%--------------------------------------


%Russian-specific packages
%--------------------------------------
\usepackage[T2A]{fontenc}
\usepackage[utf8]{inputenc} 
\usepackage[english, main=russian]{babel}
%--------------------------------------
\usepackage{textcomp}

% Красная строка
%--------------------------------------
\usepackage{indentfirst}               
%--------------------------------------             


%Graphics
%--------------------------------------
\usepackage{graphicx}
\graphicspath{ {./images/} }
\usepackage{wrapfig}
%--------------------------------------

% Полуторный интервал
%--------------------------------------
\linespread{1.3}                    
%--------------------------------------

%Выравнивание и переносы
%--------------------------------------
% Избавляемся от переполнений
\sloppy
% Запрещаем разрыв страницы после первой строки абзаца
\clubpenalty=10000
% Запрещаем разрыв страницы после последней строки абзаца
\widowpenalty=10000
%--------------------------------------

%Списки
\usepackage{enumitem}

%Подписи
\usepackage{caption} 

%Гиперссылки
\usepackage{hyperref}

\hypersetup {
	unicode=true
}

%Рисунки
%--------------------------------------
\DeclareCaptionLabelSeparator*{emdash}{~--- }
\captionsetup[figure]{labelsep=emdash,font=onehalfspacing,position=bottom}
%--------------------------------------

\usepackage{tempora}
\usepackage{amsmath}
\usepackage{color}
\usepackage{listings}
\lstset{
  belowcaptionskip=1\baselineskip,
  breaklines=true,
  frame=L,
  xleftmargin=\parindent,
  language=Python,
  showstringspaces=false,
  basicstyle=\footnotesize\ttfamily,
  keywordstyle=\bfseries\color{blue},
  commentstyle=\itshape\color{purple},
  identifierstyle=\color{black},
  stringstyle=\color{red},
}

%--------------------------------------
%			НАЧАЛО ДОКУМЕНТА
%--------------------------------------

\begin{document}

%--------------------------------------
%			ТИТУЛЬНЫЙ ЛИСТ
%--------------------------------------
\begin{titlepage}
\thispagestyle{empty}
\newpage


%Шапка титульного листа
%--------------------------------------
\vspace*{-60pt}
\hspace{-65pt}
\begin{minipage}{0.3\textwidth}
\hspace*{-20pt}\centering
\includegraphics[width=\textwidth]{emblem}
\end{minipage}
\begin{minipage}{0.67\textwidth}\small \textbf{
\vspace*{-0.7ex}
\hspace*{-6pt}\centerline{Министерство науки и высшего образования Российской Федерации}
\vspace*{-0.7ex}
\centerline{Федеральное государственное бюджетное образовательное учреждение }
\vspace*{-0.7ex}
\centerline{высшего образования}
\vspace*{-0.7ex}
\centerline{<<Московский государственный технический университет}
\vspace*{-0.7ex}
\centerline{имени Н.Э. Баумана}
\vspace*{-0.7ex}
\centerline{(национальный исследовательский университет)>>}
\vspace*{-0.7ex}
\centerline{(МГТУ им. Н.Э. Баумана)}}
\end{minipage}
%--------------------------------------

%Полосы
%--------------------------------------
\vspace{-25pt}
\hspace{-35pt}\rule{\textwidth}{2.3pt}

\vspace*{-20.3pt}
\hspace{-35pt}\rule{\textwidth}{0.4pt}
%--------------------------------------

\vspace{1.5ex}
\hspace{-35pt} \noindent \small ФАКУЛЬТЕТ\hspace{80pt} <<Информатика и системы управления>>

\vspace*{-16pt}
\hspace{47pt}\rule{0.83\textwidth}{0.4pt}

\vspace{0.5ex}
\hspace{-35pt} \noindent \small КАФЕДРА\hspace{50pt} <<Теоретическая информатика и компьютерные технологии>>

\vspace*{-16pt}
\hspace{30pt}\rule{0.866\textwidth}{0.4pt}
  
\vspace{11em}

\begin{center}
\Large {\bf Лабораторная работа № 1} \\ 
\large {\bf по курсу <<Компьютерная графика>>} \\ 
\large <<Реализация графических примитивов>>
\end{center}\normalsize

\vspace{8em}


\begin{flushright}
  {Студент группы ИУ9-42Б Волохов А. В.\hspace*{15pt} \\
  \vspace{2ex}
  Преподаватель Цалкович П. А.\hspace*{15pt}}
\end{flushright}

\bigskip

\vfill
 

\begin{center}
\textsl{Москва 2024}
\end{center}
\end{titlepage}
%--------------------------------------
%		КОНЕЦ ТИТУЛЬНОГО ЛИСТА
%--------------------------------------

\renewcommand{\ttdefault}{pcr}

\setlength{\tabcolsep}{3pt}
\newpage
\setcounter{page}{2}

\section{Задача}\label{Sect::task}
\par
Необходимо:
\newline
1) Реализовать любой графический примитив
\newline
2) Добавить любое геометрическое преобразование (сдвиг, поворот и т.д.)
\newline
3) Добавить обработку события (нажатия на кнопку и т.д.)

\section{Основная теория}
\par
В данном коде используется графический примитив - пятиугольник. Пятиугольник состоит из пяти вершин, каждая из которых определяется координатами (x, y). В данном случае, пятиугольник нарисован с помощью функции glBegin(GL POLYGON), которая указывает, что будет нарисован многоугольник.
\par
Также используется геометрическое преобразование - поворот. Поворот изображения выполняется с помощью функции glRotatef(angle, 0, 0, 1), которая поворачивает текущую матрицу на заданный угол вокруг оси Z.
\par
Используется обработка событий - нажатие клавиши. При нажатии клавиши "Пробел" увеличивается угол поворота, что приводит к повороту пятиугольника.

\pagebreak
\section{Практическая реализация}
Код представлен в Листинге 1.
\par
\begin{center}
    Листинг 1 - main.go
\end{center}

\begin{lstlisting}

import pygame
from pygame.locals import *
from OpenGL.GL import *

pygame.init()

display = (800, 600)
pygame.display.set_mode(display, DOUBLEBUF | OPENGL)


def init_pentagon():
    glBegin(GL_POLYGON)
    glColor3fv((1, 0, 0))
    glVertex2fv((0, 0.5))
    glVertex2fv((-0.4, 0))
    glVertex2fv((-0.2, -0.5))
    glVertex2fv((0.2, -0.5))
    glVertex2fv((0.4, 0))
    glEnd()


def draw_pentagon():
    glClear(GL_COLOR_BUFFER_BIT | GL_DEPTH_BUFFER_BIT)
    init_pentagon()
    pygame.display.flip()


def main():
    angle = 0
    while True:
        for event in pygame.event.get():
            if event.type == pygame.QUIT:
                pygame.quit()
                quit()
            if event.type == pygame.KEYDOWN:
                if event.key == pygame.K_SPACE:
                    angle += 10

        glMatrixMode(GL_PROJECTION)
        glLoadIdentity()
        glOrtho(-1, 1, -1, 1, -1, 1)
        glMatrixMode(GL_MODELVIEW)
        glLoadIdentity()
        glRotatef(angle, 0, 0, 1)
        draw_pentagon()


if __name__ == "__main__":
    main()
\end{lstlisting}
В результате работы программы получилcя следующий вывод:
\begin{center}
    \includegraphics[width=\linewidth]{res1.png}
    \includegraphics[width=\linewidth]{res2.png}
\end{center}
\pagebreak

\section{Заключение}
В ходе выполнения лабораторной работы была разработана программа, которая демонстрирует использование компьютерной графики для создания и отображения графических примитивов, выполнения геометрических преобразований и обработки событий.
\par
В результате работы были получены навыки работы с компьютерной графикой, геометрическими преобразованиями и обработкой событий в программах на языке Python. Эти навыки будут полезны при создании более сложных графических приложений и игр.
\end{document}