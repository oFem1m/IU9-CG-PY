\documentclass[a4paper, 14pt]{extarticle}

% Поля
%--------------------------------------
\usepackage{geometry}
\usepackage[T2A]{fontenc}
\usepackage[russian]{babel}
\usepackage{minted}
\usepackage{float}
\usepackage{graphicx} 
\geometry{a4paper,tmargin=2cm,bmargin=2cm,lmargin=3cm,rmargin=1cm}
%--------------------------------------


%Russian-specific packages
%--------------------------------------
\usepackage[T2A]{fontenc}
\usepackage[utf8]{inputenc} 
\usepackage[english, main=russian]{babel}
%--------------------------------------
\usepackage{textcomp}

% Красная строка
%--------------------------------------
\usepackage{indentfirst}               
%--------------------------------------             


%Graphics
%--------------------------------------
\usepackage{graphicx}
\graphicspath{ {./images/} }
\usepackage{wrapfig}
%--------------------------------------

% Полуторный интервал
%--------------------------------------
\linespread{1.3}                    
%--------------------------------------

%Выравнивание и переносы
%--------------------------------------
% Избавляемся от переполнений
\sloppy
% Запрещаем разрыв страницы после первой строки абзаца
\clubpenalty=10000
% Запрещаем разрыв страницы после последней строки абзаца
\widowpenalty=10000
%--------------------------------------

%Списки
\usepackage{enumitem}

%Подписи
\usepackage{caption} 

%Гиперссылки
\usepackage{hyperref}

\hypersetup {
	unicode=true
}

%Рисунки
%--------------------------------------
\DeclareCaptionLabelSeparator*{emdash}{~--- }
\captionsetup[figure]{labelsep=emdash,font=onehalfspacing,position=bottom}
%--------------------------------------

\usepackage{tempora}
\usepackage{amsmath}
\usepackage{color}
\usepackage{listings}
\lstset{
  belowcaptionskip=1\baselineskip,
  breaklines=true,
  frame=L,
  xleftmargin=\parindent,
  language=Python,
  showstringspaces=false,
  basicstyle=\footnotesize\ttfamily,
  keywordstyle=\bfseries\color{blue},
  commentstyle=\itshape\color{purple},
  identifierstyle=\color{black},
  stringstyle=\color{red},
}

%--------------------------------------
%			НАЧАЛО ДОКУМЕНТА
%--------------------------------------

\begin{document}

%--------------------------------------
%			ТИТУЛЬНЫЙ ЛИСТ
%--------------------------------------
\begin{titlepage}
\thispagestyle{empty}
\newpage


%Шапка титульного листа
%--------------------------------------
\vspace*{-60pt}
\hspace{-65pt}
\begin{minipage}{0.3\textwidth}
\hspace*{-20pt}\centering
\includegraphics[width=\textwidth]{emblem}
\end{minipage}
\begin{minipage}{0.67\textwidth}\small \textbf{
\vspace*{-0.7ex}
\hspace*{-6pt}\centerline{Министерство науки и высшего образования Российской Федерации}
\vspace*{-0.7ex}
\centerline{Федеральное государственное бюджетное образовательное учреждение }
\vspace*{-0.7ex}
\centerline{высшего образования}
\vspace*{-0.7ex}
\centerline{<<Московский государственный технический университет}
\vspace*{-0.7ex}
\centerline{имени Н.Э. Баумана}
\vspace*{-0.7ex}
\centerline{(национальный исследовательский университет)>>}
\vspace*{-0.7ex}
\centerline{(МГТУ им. Н.Э. Баумана)}}
\end{minipage}
%--------------------------------------

%Полосы
%--------------------------------------
\vspace{-25pt}
\hspace{-35pt}\rule{\textwidth}{2.3pt}

\vspace*{-20.3pt}
\hspace{-35pt}\rule{\textwidth}{0.4pt}
%--------------------------------------

\vspace{1.5ex}
\hspace{-35pt} \noindent \small ФАКУЛЬТЕТ\hspace{80pt} <<Информатика и системы управления>>

\vspace*{-16pt}
\hspace{47pt}\rule{0.83\textwidth}{0.4pt}

\vspace{0.5ex}
\hspace{-35pt} \noindent \small КАФЕДРА\hspace{50pt} <<Теоретическая информатика и компьютерные технологии>>

\vspace*{-16pt}
\hspace{30pt}\rule{0.866\textwidth}{0.4pt}
  
\vspace{11em}

\begin{center}
\Large {\bf Лабораторная работа № 6} \\
\large {\bf по курсу <<Компьютерная графика>>} \\
\large <<Построение реалистичных изображений>>
\end{center}\normalsize

\vspace{8em}


\begin{flushright}
  {Студент группы ИУ9-42Б Волохов А. В.\hspace*{15pt} \\
  \vspace{2ex}
  Преподаватель Цалкович П. А.\hspace*{15pt}}
\end{flushright}

\bigskip

\vfill


\begin{center}
\textsl{Москва 2024}
\end{center}
\end{titlepage}
%--------------------------------------
%		КОНЕЦ ТИТУЛЬНОГО ЛИСТА
%--------------------------------------

\renewcommand{\ttdefault}{pcr}

\setlength{\tabcolsep}{3pt}
\newpage
\setcounter{page}{2}

\section{Задача}\label{Sect::task}
Базовой лабораторной работой является лабораторная работа №3 (модельно-видовые
преобразования и преобразования проецирования) - реализцаия модели тора.
\par
Определить параметры модели освещения OpenGL (свойства источника света, свойства
материалов (поверхностей), характеристики глобальной модели освещения);
\par
Исследовать метод повышения реалистичности получаемых изображений сцены: влияние свойств материала поверхности (вариант А5);
\par
Реализовать алгоритм анимации: моделирование движения тела (с заданной начальной скоростью) при
условии абсолютно упругого отражения объекта от границ некоторого ограничивающего объема (вариант Б2);
\par
Реализовать наложение текстуры: использование текстуры для определения интенсивности поверхности (вариант В1).

\section{Основная теория}
\par
\textbf{Модели освещения в OpenGL:} Модель освещения в OpenGL включает несколько компонентов, позволяющих добиться реалистичного отображения объектов в 3D-пространстве. Освещение в OpenGL определяется набором источников света, которые могут быть точечными, направленными или прожекторами. Каждый источник света имеет свои свойства, такие как позиция, цвет, интенсивность и направление. Существует несколько типов освещения: окружающее (ambient), диффузное (diffuse) и зеркальное (specular). Окружающее освещение добавляет равномерный свет ко всем объектам сцены, диффузное освещение учитывает угол падения света на поверхность, а зеркальное освещение создает бликовые эффекты, имитируя отражение света от гладких поверхностей.
\par
\textbf{Свойства материалов и их влияние на реалистичность:} Свойства материалов в OpenGL играют ключевую роль в том, как объекты будут взаимодействовать с источниками света. Материалы имеют параметры, такие как диффузный цвет (diffuse), зеркальный цвет (specular), эмиссивный цвет (emissive) и коэффициент блеска (shininess). Диффузный цвет определяет основной цвет объекта, зеркальный цвет влияет на интенсивность бликов, эмиссивный цвет создает эффект самосветящегося материала, а коэффициент блеска определяет размер и резкость бликов. Комбинирование этих свойств позволяет создавать материалы, которые выглядят реалистично в различных условиях освещения.
\par
\textbf{Методы повышения реалистичности:} Для повышения реалистичности изображений сцены, важно учитывать различные методы и техники. Один из таких методов - использование текстур. Текстуры позволяют добавлять детали и вариации на поверхность объектов, имитируя сложные узоры и неоднородности, которые трудно создать с помощью только цветовых параметров. Еще один метод - анимация объектов с учетом физики. Реалистичное движение объектов, например, с учетом начальной скорости и упругих столкновений с границами, добавляет динамику и правдоподобность сцене.
\par
\textbf{Анимация и моделирование движения:} Анимация в OpenGL позволяет создавать движущиеся объекты, что добавляет реалистичность и динамику в сцены. В данном контексте моделирование движения тела с заданной начальной скоростью и условием абсолютно упругого отражения от границ объема представляет собой простую физическую симуляцию. При этом объект изменяет свою позицию в пространстве с течением времени, а при столкновении с границами сцены его скорость изменяет направление, имитируя упругое столкновение. Такой подход позволяет наблюдать за естественным движением объектов и их взаимодействием с окружением.

\pagebreak
\section{Практическая реализация}
Код представлен в Листинге 1.
\par
\begin{center}
    Листинг 1 - lab6.py
\end{center}

\begin{lstlisting}
import math

import glfw
import numpy as np
from OpenGL.GL import *
from math import cos, sin

from PIL import Image

alpha = 0
beta = 0
size = 0.5
fill = True

torus_position = [0.0, 0.0, 0.0]
torus_velocity = [0.0001, 0.0002, 0.0005]


def main():
    if not glfw.init():
        return
    window = glfw.create_window(640, 640, "LAB 6", None, None)
    if not window:
        glfw.terminate()
        return
    glfw.make_context_current(window)
    glfw.set_key_callback(window, key_callback)
    glfw.set_scroll_callback(window, scroll_callback)

    glEnable(GL_DEPTH_TEST)
    glDepthFunc(GL_LESS)

    setup_lighting()
    texture_id = load_texture("texture.jpg")

    while not glfw.window_should_close(window):
        display(window, texture_id)
    glfw.destroy_window(window)
    glfw.terminate()


def setup_lighting():
    glEnable(GL_LIGHTING)
    glEnable(GL_LIGHT0)

    light_pos = [10.0, 10.0, 10.0, 1.0]
    glLightfv(GL_LIGHT0, GL_POSITION, light_pos)

    light_diffuse = [1.0, 1.0, 1.0, 1.0]
    glLightfv(GL_LIGHT0, GL_DIFFUSE, light_diffuse)
    light_specular = [1.0, 1.0, 1.0, 1.0]
    glLightfv(GL_LIGHT0, GL_SPECULAR, light_specular)

    material_diffuse = [0.8, 0.8, 0.8, 1.0]
    glMaterialfv(GL_FRONT, GL_DIFFUSE, material_diffuse)
    material_specular = [1.0, 1.0, 1.0, 1.0]
    glMaterialfv(GL_FRONT, GL_SPECULAR, material_specular)
    material_shininess = [100.0]
    glMaterialfv(GL_FRONT, GL_SHININESS, material_shininess)

def load_texture(filename):
    img = Image.open(filename)
    img_data = np.array(list(img.getdata()), np.uint8)
    texture_id = glGenTextures(1)
    glBindTexture(GL_TEXTURE_2D, texture_id)
    glTexParameteri(GL_TEXTURE_2D, GL_TEXTURE_MIN_FILTER, GL_LINEAR)
    glTexParameteri(GL_TEXTURE_2D, GL_TEXTURE_MAG_FILTER, GL_LINEAR)
    glTexImage2D(GL_TEXTURE_2D, 0, GL_RGB, img.width, img.height, 0, GL_RGB, GL_UNSIGNED_BYTE, img_data)
    return texture_id


def display(window, texture_id):
    global alpha
    global beta
    global size
    global torus_position
    global torus_velocity

    glLoadIdentity()
    glClear(GL_COLOR_BUFFER_BIT)
    glClear(GL_DEPTH_BUFFER_BIT)
    glMatrixMode(GL_PROJECTION)

    def projection():
        alpha_rad = np.radians(alpha)
        beta_rad = np.radians(beta)

        rotate_y = np.array([
            [cos(alpha_rad), 0, sin(alpha_rad), 0],
            [0, 1, 0, 0],
            [-sin(alpha_rad), 0, cos(alpha_rad), 0],
            [0, 0, 0, 1]
        ])

        rotate_x = np.array([
            [1, 0, 0, 0],
            [0, cos(beta_rad), -sin(beta_rad), 0],
            [0, sin(beta_rad), cos(beta_rad), 0],
            [0, 0, 0, 1]
        ])

        glMultMatrixf(rotate_x)
        glMultMatrixf(rotate_y)

    def torus(R, r, N, n):
        glEnable(GL_TEXTURE_2D)
        glBindTexture(GL_TEXTURE_2D, texture_id)

        for i in range(N):
            for j in range(n):
                theta = (2 * math.pi / N) * i
                phi = (2 * math.pi / n) * j
                theta_next = (2 * math.pi / N) * (i + 1)
                phi_next = (2 * math.pi / n) * (j + 1)

                x0 = (R + r * cos(phi)) * cos(theta)
                y0 = (R + r * cos(phi)) * sin(theta)
                z0 = r * sin(phi)

                x1 = (R + r * cos(phi)) * cos(theta_next)
                y1 = (R + r * cos(phi)) * sin(theta_next)
                z1 = r * sin(phi)

                x2 = (R + r * cos(phi_next)) * cos(theta_next)
                y2 = (R + r * cos(phi_next)) * sin(theta_next)
                z2 = r * sin(phi_next)

                x3 = (R + r * cos(phi_next)) * cos(theta)
                y3 = (R + r * cos(phi_next)) * sin(theta)
                z3 = r * sin(phi_next)

                glBegin(GL_QUADS)

                glTexCoord2f(0.0, 0.0)
                glVertex3f(x0, y0, z0)

                glTexCoord2f(1.0, 0.0)
                glVertex3f(x1, y1, z1)

                glTexCoord2f(1.0, 1.0)
                glVertex3f(x2, y2, z2)

                glTexCoord2f(0.0, 1.0)
                glVertex3f(x3, y3, z3)

                glEnd()

        glDisable(GL_TEXTURE_2D)

    glLoadIdentity()

    projection()
    R = size
    r = size / 3

    for i in range(3):
        torus_position[i] += torus_velocity[i]

    for i in range(3):
        if torus_position[i] + size > 1.0 or torus_position[i] - size < -1.0:
            torus_velocity[i] *= -1.0

    glTranslatef(torus_position[0], torus_position[1], torus_position[2])
    torus(R, r, 40, 25)

    glfw.swap_buffers(window)
    glfw.poll_events()


def key_callback(window, key, scancode, action, mods):
    global alpha
    global beta

    if action == glfw.PRESS or action == glfw.REPEAT:
        if key == glfw.KEY_RIGHT:
            alpha += 3
        elif key == glfw.KEY_LEFT:
            alpha -= 3
        elif key == glfw.KEY_UP:
            beta -= 3
        elif key == glfw.KEY_DOWN:
            beta += 3
        elif key == glfw.KEY_F:
            global fill
            fill = not fill
            if fill:
                glPolygonMode(GL_FRONT_AND_BACK, GL_FILL)
            else:
                glPolygonMode(GL_FRONT_AND_BACK, GL_LINE)
        elif key == glfw.KEY_L and action == glfw.PRESS:
            light_enabled = glIsEnabled(GL_LIGHT0)
            if light_enabled:
                glDisable(GL_LIGHT0)
            else:
                glEnable(GL_LIGHT0)


def scroll_callback(window, xoffset, yoffset):
    global size

    if xoffset > 0:
        size -= yoffset / 10
    else:
        size += yoffset / 10


if __name__ == "__main__":
    main()
\end{lstlisting}
\pagebreak
В результате работы программы получилcя следующий вывод:
\begin{center}
    \includegraphics[width=\linewidth]{res1.png}
    \newpage
    \includegraphics[width=\linewidth]{res2.png}
    \newpage
    \includegraphics[width=\linewidth]{res3.png}
\end{center}
\pagebreak

\section{Заключение}
\par
В данной лабораторной работе были реализованы различные аспекты создания реалистичных изображений с использованием OpenGL. Определены и настроены параметры модели освещения, исследованы свойства материалов, их влияние на внешний вид объектов, а также использованы текстуры для повышения реалистичности поверхности. Кроме того, был реализован алгоритм анимации, моделирующий движение объекта с учетом абсолютно упругого отражения от границ объема. Эти методы и техники в совокупности позволяют создавать визуально правдоподобные сцены, что является важным в изучении компьютерной графики.
\end{document}