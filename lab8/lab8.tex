\documentclass[a4paper, 14pt]{extarticle}

% Поля
%--------------------------------------
\usepackage{geometry}
\usepackage[T2A]{fontenc}
\usepackage[russian]{babel}
\usepackage{minted}
\usepackage{float}
\usepackage{graphicx} 
\geometry{a4paper,tmargin=2cm,bmargin=2cm,lmargin=3cm,rmargin=1cm}
%--------------------------------------


%Russian-specific packages
%--------------------------------------
\usepackage[T2A]{fontenc}
\usepackage[utf8]{inputenc} 
\usepackage[english, main=russian]{babel}
%--------------------------------------
\usepackage{textcomp}

% Красная строка
%--------------------------------------
\usepackage{indentfirst}               
%--------------------------------------             


%Graphics
%--------------------------------------
\usepackage{graphicx}
\graphicspath{ {./images/} }
\usepackage{wrapfig}
%--------------------------------------

% Полуторный интервал
%--------------------------------------
\linespread{1.3}                    
%--------------------------------------

%Выравнивание и переносы
%--------------------------------------
% Избавляемся от переполнений
\sloppy
% Запрещаем разрыв страницы после первой строки абзаца
\clubpenalty=10000
% Запрещаем разрыв страницы после последней строки абзаца
\widowpenalty=10000
%--------------------------------------

%Списки
\usepackage{enumitem}

%Подписи
\usepackage{caption} 

%Гиперссылки
\usepackage{hyperref}

\hypersetup {
	unicode=true
}

%Рисунки
%--------------------------------------
\DeclareCaptionLabelSeparator*{emdash}{~--- }
\captionsetup[figure]{labelsep=emdash,font=onehalfspacing,position=bottom}
%--------------------------------------

\usepackage{tempora}
\usepackage{amsmath}
\usepackage{color}
\usepackage{listings}
\lstset{
  belowcaptionskip=1\baselineskip,
  breaklines=true,
  frame=L,
  xleftmargin=\parindent,
  language=Python,
  showstringspaces=false,
  basicstyle=\footnotesize\ttfamily,
  keywordstyle=\bfseries\color{blue},
  commentstyle=\itshape\color{purple},
  identifierstyle=\color{black},
  stringstyle=\color{red},
}

%--------------------------------------
%			НАЧАЛО ДОКУМЕНТА
%--------------------------------------

\begin{document}

%--------------------------------------
%			ТИТУЛЬНЫЙ ЛИСТ
%--------------------------------------
\begin{titlepage}
\thispagestyle{empty}
\newpage


%Шапка титульного листа
%--------------------------------------
\vspace*{-60pt}
\hspace{-65pt}
\begin{minipage}{0.3\textwidth}
\hspace*{-20pt}\centering
\includegraphics[width=\textwidth]{emblem}
\end{minipage}
\begin{minipage}{0.67\textwidth}\small \textbf{
\vspace*{-0.7ex}
\hspace*{-6pt}\centerline{Министерство науки и высшего образования Российской Федерации}
\vspace*{-0.7ex}
\centerline{Федеральное государственное бюджетное образовательное учреждение }
\vspace*{-0.7ex}
\centerline{высшего образования}
\vspace*{-0.7ex}
\centerline{<<Московский государственный технический университет}
\vspace*{-0.7ex}
\centerline{имени Н.Э. Баумана}
\vspace*{-0.7ex}
\centerline{(национальный исследовательский университет)>>}
\vspace*{-0.7ex}
\centerline{(МГТУ им. Н.Э. Баумана)}}
\end{minipage}
%--------------------------------------

%Полосы
%--------------------------------------
\vspace{-25pt}
\hspace{-35pt}\rule{\textwidth}{2.3pt}

\vspace*{-20.3pt}
\hspace{-35pt}\rule{\textwidth}{0.4pt}
%--------------------------------------

\vspace{1.5ex}
\hspace{-35pt} \noindent \small ФАКУЛЬТЕТ\hspace{80pt} <<Информатика и системы управления>>

\vspace*{-16pt}
\hspace{47pt}\rule{0.83\textwidth}{0.4pt}

\vspace{0.5ex}
\hspace{-35pt} \noindent \small КАФЕДРА\hspace{50pt} <<Теоретическая информатика и компьютерные технологии>>

\vspace*{-16pt}
\hspace{30pt}\rule{0.866\textwidth}{0.4pt}
  
\vspace{11em}

\begin{center}
\Large {\bf Лабораторная работа №8 } \\
\large {\bf по курсу <<Компьютерная графика>>} \\
\large <<Шейдеры OpenGL>>
\end{center}\normalsize

\vspace{8em}


\begin{flushright}
  {Студент группы ИУ9-42Б Волохов А. В.\hspace*{15pt} \\
  \vspace{2ex}
  Преподаватель Цалкович П. А.\hspace*{15pt}}
\end{flushright}

\bigskip

\vfill


\begin{center}
\textsl{Москва 2024}
\end{center}
\end{titlepage}
%--------------------------------------
%		КОНЕЦ ТИТУЛЬНОГО ЛИСТА
%--------------------------------------

\renewcommand{\ttdefault}{pcr}

\setlength{\tabcolsep}{3pt}
\newpage
\setcounter{page}{2}

\section{Условие}\label{Sect::task}
\par
Задача: Переписать на шейдерах одну из лабораторных: 2, 3 или 6. Была выбрана лаболраторная работа 6.
\section{Основная теория}
\par
\textbf{OpenGL и шейдеры:} OpenGL — это мощная библиотека для работы с 2D и 3D графикой, которая предоставляет API для взаимодействия с графическим процессором (GPU) для создания высокоэффективных графических приложений. Одним из ключевых элементов OpenGL являются шейдеры — небольшие программы, которые выполняются на GPU и позволяют разработчикам управлять процессом рендеринга на низком уровне.
\par
Шейдеры в OpenGL делятся на несколько типов, но основные из них — это вершинные и фрагментные шейдеры. Вершинные шейдеры обрабатывают вершины, определяя их положение в пространстве, а фрагментные шейдеры определяют цвет каждого пикселя. Шейдеры пишутся на языке GLSL (OpenGL Shading Language), который предоставляет возможности для выполнения математических операций, необходимых для графических вычислений.
\par
В этой лаборатной работе использовались два типа шейдеров: вершинный и фрагментный. Вершинный шейдер преобразует координаты вершин модели из локальной системы координат в экранные координаты, а также передает необходимые данные во фрагментный шейдер, такие как нормали и текстурные координаты. Фрагментный шейдер выполняет освещение с использованием модели фонового, диффузного и зеркального освещения, а также применяет текстуру к поверхности объекта.
\par
Для создания шейдерной программы в OpenGL необходимо скомпилировать исходные коды шейдеров и связать их в единую программу, которая затем используется в процессе рендеринга. В программе, описанной ниже, это выполняется функциями \texttt{compile\_shader} и \texttt{create\_shader\_program}.
\par
Для текстурирования объекта была использована текстура, загружаемая с помощью библиотеки PIL и применяемая к объекту через шейдеры.

\pagebreak
\section{Практическая реализация}
Код представлен в Листинге 1.
\par
\begin{center}
    Листинг 1 - lab7.py
\end{center}

\begin{lstlisting}
import math

import glfw
import numpy as np
from OpenGL.GL import *
from math import cos, sin

from PIL import Image

alpha = 0
beta = 0
size = 0.5
fill = True

torus_position = [0.0, 0.0, 0.0]
torus_velocity = [0.001, 0.002, 0.005]


def main():
    if not glfw.init():
        return
    window = glfw.create_window(640, 640, "LAB 8", None, None)
    if not window:
        glfw.terminate()
        return
    glfw.make_context_current(window)
    glfw.set_key_callback(window, key_callback)
    glfw.set_scroll_callback(window, scroll_callback)

    glEnable(GL_DEPTH_TEST)
    glDepthFunc(GL_LESS)

    setup_lighting()
    texture_id = load_texture("texture.jpg")

    vertex_shader_source = """
        attribute vec3 aVert;
        varying vec3 n;
        varying vec3 v;
        varying vec2 uv;
        void main()
        {
            uv = gl_MultiTexCoord0.xy;
            v = vec3(gl_ModelViewMatrix * vec4(aVert, 1.0));
            n = normalize(gl_NormalMatrix * gl_Normal);
            gl_TexCoord[0] = gl_TextureMatrix[0] * gl_MultiTexCoord0;
            gl_Position = gl_ModelViewProjectionMatrix * vec4(aVert, 1.0);
        }
    """

    fragment_shader_source = """
        varying vec3 n;
        varying vec3 v;
        uniform sampler2D tex;
        void main ()
        {
            vec3 L = normalize(gl_LightSource[0].position.xyz - v);
            vec3 E = normalize(-v);
            vec3 R = normalize(-reflect(L,n));

            vec4 Iamb = gl_FrontLightProduct[0].ambient;
            vec4 Idiff = gl_FrontLightProduct[0].diffuse * max(dot(n,L), 0.0);
            Idiff = clamp(Idiff, 0.0, 1.0);

            vec4 Ispec = gl_LightSource[0].specular
                        * pow(max(dot(R,E),0.0),0.7);
            Ispec = clamp(Ispec, 0.0, 1.0);

            vec4 texColor = texture2D(tex, gl_TexCoord[0].st);
            gl_FragColor = (Idiff + Iamb + Ispec) * texColor;
        }
    """

    shader_program = create_shader_program(vertex_shader_source, fragment_shader_source)
    glUseProgram(shader_program)

    while not glfw.window_should_close(window):
        display(window, texture_id, shader_program)
    glfw.destroy_window(window)
    glfw.terminate()


def setup_lighting():
    glEnable(GL_LIGHTING)
    glEnable(GL_LIGHT0)

    light_pos = [10.0, 10.0, 10.0, 1.0]
    glLightfv(GL_LIGHT0, GL_POSITION, light_pos)

    light_diffuse = [1.0, 1.0, 1.0, 1.0]
    glLightfv(GL_LIGHT0, GL_DIFFUSE, light_diffuse)
    light_specular = [1.0, 1.0, 1.0, 1.0]
    glLightfv(GL_LIGHT0, GL_SPECULAR, light_specular)

    material_diffuse = [0.8, 0.8, 0.8, 1.0]
    glMaterialfv(GL_FRONT, GL_DIFFUSE, material_diffuse)
    material_specular = [1.0, 1.0, 1.0, 1.0]
    glMaterialfv(GL_FRONT, GL_SPECULAR, material_specular)
    material_shininess = [100.0]
    glMaterialfv(GL_FRONT, GL_SHININESS, material_shininess)


def load_texture(filename):
    img = Image.open(filename)
    img_data = np.array(list(img.getdata()), np.uint8)
    texture_id = glGenTextures(1)
    glBindTexture(GL_TEXTURE_2D, texture_id)
    glTexParameteri(GL_TEXTURE_2D, GL_TEXTURE_MIN_FILTER, GL_LINEAR)
    glTexParameteri(GL_TEXTURE_2D, GL_TEXTURE_MAG_FILTER, GL_LINEAR)
    glTexImage2D(GL_TEXTURE_2D, 0, GL_RGB, img.width, img.height, 0, GL_RGB, GL_UNSIGNED_BYTE, img_data)
    return texture_id


def compile_shader(source, shader_type):
    shader = glCreateShader(shader_type)
    glShaderSource(shader, source)
    glCompileShader(shader)

    if glGetShaderiv(shader, GL_COMPILE_STATUS) != GL_TRUE:
        raise RuntimeError(glGetShaderInfoLog(shader))

    return shader


def create_shader_program(vertex_source, fragment_source):
    program = glCreateProgram()
    vertex_shader = compile_shader(vertex_source, GL_VERTEX_SHADER)
    fragment_shader = compile_shader(fragment_source, GL_FRAGMENT_SHADER)

    glAttachShader(program, vertex_shader)
    glAttachShader(program, fragment_shader)
    glLinkProgram(program)

    if glGetProgramiv(program, GL_LINK_STATUS) != GL_TRUE:
        raise RuntimeError(glGetProgramInfoLog(program))

    glDeleteShader(vertex_shader)
    glDeleteShader(fragment_shader)

    return program


def display(window, texture_id, shader_program):
    global alpha
    global beta
    global size
    global torus_position
    global torus_velocity

    glLoadIdentity()
    glClear(GL_COLOR_BUFFER_BIT)
    glClear(GL_DEPTH_BUFFER_BIT)
    glMatrixMode(GL_PROJECTION)

    alpha_rad = np.radians(alpha)
    beta_rad = np.radians(beta)

    rotate_y = np.array([
        [cos(alpha_rad), 0, sin(alpha_rad), 0],
        [0, 1, 0, 0],
        [-sin(alpha_rad), 0, cos(alpha_rad), 0],
        [0, 0, 0, 1]
    ], dtype=np.float32)

    rotate_x = np.array([
        [1, 0, 0, 0],
        [0, cos(beta_rad), -sin(beta_rad), 0],
        [0, sin(beta_rad), cos(beta_rad), 0],
        [0, 0, 0, 1]
    ], dtype=np.float32)

    projection = np.identity(4, dtype=np.float32)
    model = np.dot(rotate_x, rotate_y)
    view = np.identity(4, dtype=np.float32)

    R = size
    r = size / 3

    for i in range(3):
        torus_position[i] += torus_velocity[i]

    for i in range(3):
        if torus_position[i] + size > 1.0 or torus_position[i] - size < -1.0:
            torus_velocity[i] *= -1.0

    model = np.dot(model, np.array([
        [1, 0, 0, torus_position[0]],
        [0, 1, 0, torus_position[1]],
        [0, 0, 1, torus_position[2]],
        [0, 0, 0, 1]
        ], dtype=np.float32))

    model_loc = glGetUniformLocation(shader_program, "model")
    view_loc = glGetUniformLocation(shader_program, "view")
    projection_loc = glGetUniformLocation(shader_program, "projection")

    glUniformMatrix4fv(model_loc, 1, GL_TRUE, model)
    glUniformMatrix4fv(view_loc, 1, GL_TRUE, view)
    glUniformMatrix4fv(projection_loc, 1, GL_TRUE, projection)

    torus(R, r, 40, 25, shader_program, texture_id)

    glfw.swap_buffers(window)
    glfw.poll_events()

def torus(R, r, N, n, shader_program, texture_id):
    glEnable(GL_TEXTURE_2D)
    glBindTexture(GL_TEXTURE_2D, texture_id)
    vertices = []
    tex_coords = []

    for i in range(N):
        for j in range(n):
            theta = (2 * math.pi / N) * i
            phi = (2 * math.pi / n) * j
            theta_next = (2 * math.pi / N) * (i + 1)
            phi_next = (2 * math.pi / n) * (j + 1)

            x0 = (R + r * cos(phi)) * cos(theta)
            y0 = (R + r * cos(phi)) * sin(theta)
            z0 = r * sin(phi)
            vertices.extend([x0, y0, z0])
            tex_coords.extend([j / n, i / N])

            x1 = (R + r * cos(phi)) * cos(theta_next)
            y1 = (R + r * cos(phi)) * sin(theta_next)
            z1 = r * sin(phi)
            vertices.extend([x1, y1, z1])
            tex_coords.extend([(j + 1) / n, i / N])

            x2 = (R + r * cos(phi_next)) * cos(theta_next)
            y2 = (R + r * cos(phi_next)) * sin(theta_next)
            z2 = r * sin(phi_next)
            vertices.extend([x2, y2, z2])
            tex_coords.extend([(j + 1) / n, (i + 1) / N])

            x3 = (R + r * cos(phi_next)) * cos(theta)
            y3 = (R + r * cos(phi_next)) * sin(theta)
            z3 = r * sin(phi_next)
            vertices.extend([x3, y3, z3])
            tex_coords.extend([j / n, (i + 1) / N])

    vertices = np.array(vertices, dtype=np.float32)
    tex_coords = np.array(tex_coords, dtype=np.float32)

    vao = glGenVertexArrays(1)
    vbo = glGenBuffers(2)

    glBindVertexArray(vao)

    glBindBuffer(GL_ARRAY_BUFFER, vbo[0])
    glBufferData(GL_ARRAY_BUFFER, vertices.nbytes, vertices, GL_STATIC_DRAW)
    glVertexAttribPointer(0, 3, GL_FLOAT, GL_FALSE, 0, None)
    glEnableVertexAttribArray(0)

    glBindBuffer(GL_ARRAY_BUFFER, vbo[1])
    glBufferData(GL_ARRAY_BUFFER, tex_coords.nbytes, tex_coords, GL_STATIC_DRAW)
    glVertexAttribPointer(1, 2, GL_FLOAT, GL_FALSE, 0, None)
    glEnableVertexAttribArray(1)

    glBindVertexArray(vao)
    glDrawArrays(GL_QUADS, 0, len(vertices) // 3)

    glDisable(GL_TEXTURE_2D)
    glBindVertexArray(0)


def key_callback(window, key, scancode, action, mods):
    global alpha
    global beta

    if action == glfw.PRESS or action == glfw.REPEAT:
        if key == glfw.KEY_RIGHT:
            alpha += 3
        elif key == glfw.KEY_LEFT:
            alpha -= 3
        elif key == glfw.KEY_UP:
            beta -= 3
        elif key == glfw.KEY_DOWN:
            beta += 3
        elif key == glfw.KEY_F:
            global fill
            fill = not fill
            if fill:
                glPolygonMode(GL_FRONT_AND_BACK, GL_FILL)
            else:
                glPolygonMode(GL_FRONT_AND_BACK, GL_LINE)
        elif key == glfw.KEY_L and action == glfw.PRESS:
            light_enabled = glIsEnabled(GL_LIGHT0)
            if light_enabled:
                glDisable(GL_LIGHT0)
            else:
                glEnable(GL_LIGHT0)

def scroll_callback(window, xoffset, yoffset):
    global size

    if xoffset > 0:
        size -= yoffset / 10
    else:
        size += yoffset / 10


if __name__ == "__main__":
    main()
\end{lstlisting}
\pagebreak
В результате работы программы получилcя следующий вывод:
\begin{center}
    \includegraphics[width=\linewidth]{res1.png}
    \newpage
\end{center}
\pagebreak
\section{Заключение}
\par
Реализация шейдеров в OpenGL позволяет существенно улучшить качество графики и производительность приложения. В данном проекте были успешно применены вершинные и фрагментные шейдеры для реализации анимации текстурированного тора с освещением. Шейдеры предоставляют гибкие и мощные инструменты для управления графическим процессом, что делает их неотъемлемой частью современной компьютерной графики.
\end{document}