\documentclass[a4paper, 14pt]{extarticle}

% Поля
%--------------------------------------
\usepackage{geometry}
\usepackage[T2A]{fontenc}
\usepackage[russian]{babel}
\usepackage{minted}
\usepackage{float}
\usepackage{graphicx} 
\geometry{a4paper,tmargin=2cm,bmargin=2cm,lmargin=3cm,rmargin=1cm}
%--------------------------------------


%Russian-specific packages
%--------------------------------------
\usepackage[T2A]{fontenc}
\usepackage[utf8]{inputenc} 
\usepackage[english, main=russian]{babel}
%--------------------------------------
\usepackage{textcomp}

% Красная строка
%--------------------------------------
\usepackage{indentfirst}               
%--------------------------------------             


%Graphics
%--------------------------------------
\usepackage{graphicx}
\graphicspath{ {./images/} }
\usepackage{wrapfig}
%--------------------------------------

% Полуторный интервал
%--------------------------------------
\linespread{1.3}                    
%--------------------------------------

%Выравнивание и переносы
%--------------------------------------
% Избавляемся от переполнений
\sloppy
% Запрещаем разрыв страницы после первой строки абзаца
\clubpenalty=10000
% Запрещаем разрыв страницы после последней строки абзаца
\widowpenalty=10000
%--------------------------------------

%Списки
\usepackage{enumitem}

%Подписи
\usepackage{caption} 

%Гиперссылки
\usepackage{hyperref}

\hypersetup {
	unicode=true
}

%Рисунки
%--------------------------------------
\DeclareCaptionLabelSeparator*{emdash}{~--- }
\captionsetup[figure]{labelsep=emdash,font=onehalfspacing,position=bottom}
%--------------------------------------

\usepackage{tempora}
\usepackage{amsmath}
\usepackage{color}
\usepackage{listings}
\lstset{
  belowcaptionskip=1\baselineskip,
  breaklines=true,
  frame=L,
  xleftmargin=\parindent,
  language=Python,
  showstringspaces=false,
  basicstyle=\footnotesize\ttfamily,
  keywordstyle=\bfseries\color{blue},
  commentstyle=\itshape\color{purple},
  identifierstyle=\color{black},
  stringstyle=\color{red},
}

%--------------------------------------
%			НАЧАЛО ДОКУМЕНТА
%--------------------------------------

\begin{document}

%--------------------------------------
%			ТИТУЛЬНЫЙ ЛИСТ
%--------------------------------------
\begin{titlepage}
\thispagestyle{empty}
\newpage


%Шапка титульного листа
%--------------------------------------
\vspace*{-60pt}
\hspace{-65pt}
\begin{minipage}{0.3\textwidth}
\hspace*{-20pt}\centering
\includegraphics[width=\textwidth]{emblem}
\end{minipage}
\begin{minipage}{0.67\textwidth}\small \textbf{
\vspace*{-0.7ex}
\hspace*{-6pt}\centerline{Министерство науки и высшего образования Российской Федерации}
\vspace*{-0.7ex}
\centerline{Федеральное государственное бюджетное образовательное учреждение }
\vspace*{-0.7ex}
\centerline{высшего образования}
\vspace*{-0.7ex}
\centerline{<<Московский государственный технический университет}
\vspace*{-0.7ex}
\centerline{имени Н.Э. Баумана}
\vspace*{-0.7ex}
\centerline{(национальный исследовательский университет)>>}
\vspace*{-0.7ex}
\centerline{(МГТУ им. Н.Э. Баумана)}}
\end{minipage}
%--------------------------------------

%Полосы
%--------------------------------------
\vspace{-25pt}
\hspace{-35pt}\rule{\textwidth}{2.3pt}

\vspace*{-20.3pt}
\hspace{-35pt}\rule{\textwidth}{0.4pt}
%--------------------------------------

\vspace{1.5ex}
\hspace{-35pt} \noindent \small ФАКУЛЬТЕТ\hspace{80pt} <<Информатика и системы управления>>

\vspace*{-16pt}
\hspace{47pt}\rule{0.83\textwidth}{0.4pt}

\vspace{0.5ex}
\hspace{-35pt} \noindent \small КАФЕДРА\hspace{50pt} <<Теоретическая информатика и компьютерные технологии>>

\vspace*{-16pt}
\hspace{30pt}\rule{0.866\textwidth}{0.4pt}
  
\vspace{11em}

\begin{center}
\Large {\bf Лабораторная работа № 7} \\
\large {\bf по курсу <<Компьютерная графика>>} \\
\large <<Оптимизация приложений OpenGL>>
\end{center}\normalsize

\vspace{8em}


\begin{flushright}
  {Студент группы ИУ9-42Б Волохов А. В.\hspace*{15pt} \\
  \vspace{2ex}
  Преподаватель Цалкович П. А.\hspace*{15pt}}
\end{flushright}

\bigskip

\vfill


\begin{center}
\textsl{Москва 2024}
\end{center}
\end{titlepage}
%--------------------------------------
%		КОНЕЦ ТИТУЛЬНОГО ЛИСТА
%--------------------------------------

\renewcommand{\ttdefault}{pcr}

\setlength{\tabcolsep}{3pt}
\newpage
\setcounter{page}{2}

\section{Условие}\label{Sect::task}
Целью данной лабораторной работы является изучение эффективных приемов организации приложений и оптимизации вызовов OpenGL.
\par
Задача заключается в оптимизации приложения OpenGL, созданного в рамках предыдущей лабораторной работы, на основе выбора наиболее эффективных методик с целью повышения FPS.
\par
Необходимо обязательно использовать дисплейные списки и массивы вершин, а также еще две любые различные оптимизации (в сумме минимум 4 оптимизации).
\parОценка применимости выбранного метода оптимизации должна осуществляться на основании измерения производительности. Результаты замеров следует оформить в табличном виде.

\section{Основная теория}
\par
\textbf{OpenGL и его оптимизация:} OpenGL (Open Graphics Library) - это спецификация, определяющая кросс-платформенный API для рендеринга 2D и 3D графики. Она предоставляет разработчикам мощные инструменты для создания графических приложений, таких как игры, симуляции и научные визуализации. Однако для достижения высокой производительности требуется эффективная организация и оптимизация вызовов OpenGL. Существует множество методик, позволяющих улучшить производительность графических приложений.
\par
\textbf{Дисплейные списки:} Дисплейные списки являются одним из наиболее простых и эффективных методов оптимизации. Они позволяют заранее компилировать и сохранять последовательности команд OpenGL, которые затем можно многократно выполнять с помощью одного вызова. Это существенно снижает нагрузку на процессор и ускоряет рендеринг, особенно для сложных геометрических объектов, которые необходимо часто отрисовывать.
\par
\textbf{Массивы вершин и буферы вершин:} Использование массивов вершин и буферов вершин (Vertex Buffer Objects, VBOs) позволяет хранить геометрические данные в видеопамяти, что значительно уменьшает накладные расходы на передачу данных между центральным процессором и графическим процессором. Это особенно полезно для рендеринга сложных сцен с большим количеством вершин и примитивов. Буферы вершин обеспечивают высокую производительность за счет минимизации количества вызовов функций и более эффективного использования аппаратных ресурсов.
\par
\textbf{Оптимизация источников света:} Правильная настройка источников света также может значительно повлиять на производительность. Например, установка позиции источника света с использованием координат в форме (x, y, z, 0) позволяет упростить расчеты и повысить эффективность рендеринга. Это особенно важно для сцен с множеством источников света, где каждый дополнительный источник может значительно увеличить вычислительную нагрузку.
\par
\textbf{Использование эффективных форматов хранения изображений:} При работе с текстурами выбор правильного формата хранения изображений может существенно повлиять на производительность. Форматы, такие как GL UNSIGNED BYTE, обеспечивают быструю загрузку и обработку текстурных данных, что особенно важно для интерактивных приложений с высокой частотой кадров. Это позволяет ускорить процессы загрузки текстур в видеопамять и их последующее использование при рендеринге.
\pagebreak
\section{Практическая реализация}
Код представлен в Листинге 1.
\par
\begin{center}
    Листинг 1 - lab7.py
\end{center}

\begin{lstlisting}
import math
import time

import glfw
import numpy as np
from OpenGL.GL import *
from math import cos, sin

from PIL import Image

alpha = 0
beta = 0
size = 0.5
fill = True

torus_position = [0.0, 0.0, 0.0]
torus_velocity = [0.0000000001, 0.0000000002, 0.0000000005]

last_frame_time = 0.0
frame_count = 0
fps = 0.0

torus_display_list = None
vbo = None


def create_torus_display_list():
    global torus_display_list

    torus_display_list = glGenLists(1)
    glNewList(torus_display_list, GL_COMPILE)

    R = size
    r = size / 3
    N = 40
    n = 25

    for i in range(N):
        glBegin(GL_QUAD_STRIP)
        for j in range(n + 1):
            for k in range(2):
                s = (i + k) % N + 0.5
                t = j % (n + 1)
                theta = 2 * math.pi * s / N
                phi = 2 * math.pi * t / n
                x = (R + r * cos(phi)) * cos(theta)
                y = (R + r * cos(phi)) * sin(theta)
                z = r * sin(phi)
                glTexCoord2f(s / N, t / n)
                glVertex3f(x, y, z)
        glEnd()

    glEndList()


vertices = []


def create_vbo():
    global vbo

    R = size
    r = size / 3
    N = 40
    n = 25

    for i in range(N):
        for j in range(n + 1):
            for k in range(2):
                s = (i + k) % N + 0.5
                t = j % (n + 1)
                theta = 2 * math.pi * s / N
                phi = 2 * math.pi * t / n
                x = (R + r * cos(phi)) * cos(theta)
                y = (R + r * cos(phi)) * sin(theta)
                z = r * sin(phi)
                vertices.extend([x, y, z])

    vbo = glGenBuffers(1)
    glBindBuffer(GL_ARRAY_BUFFER, vbo)
    glBufferData(GL_ARRAY_BUFFER, np.array(vertices, dtype=np.float32), GL_STATIC_DRAW)


def draw_torus_with_vbo():
    global vertices

    glBindBuffer(GL_ARRAY_BUFFER, vbo)
    glEnableClientState(GL_VERTEX_ARRAY)
    glVertexPointer(3, GL_FLOAT, 0, None)
    glDrawArrays(GL_QUAD_STRIP, 0, int(len(vertices) / 3))
    glDisableClientState(GL_VERTEX_ARRAY)


def main():
    global last_frame_time, frame_count, fps
    if not glfw.init():
        return
    window = glfw.create_window(640, 640, "LAB 7 opt", None, None)
    if not window:
        glfw.terminate()
        return
    glfw.make_context_current(window)
    glfw.set_key_callback(window, key_callback)
    glfw.set_scroll_callback(window, scroll_callback)

    glEnable(GL_DEPTH_TEST)
    glDepthFunc(GL_LESS)

    setup_lighting()
    texture_id = load_texture("texture.jpg")

    create_torus_display_list()
    create_vbo()

    last_frame_time = time.time()

    while not glfw.window_should_close(window):
        display(window, texture_id)

    glfw.destroy_window(window)
    glfw.terminate()


def setup_lighting():
    glEnable(GL_LIGHTING)
    glEnable(GL_LIGHT0)

    light_pos = [10.0, 10.0, 10.0, 0.0]
    glLightfv(GL_LIGHT0, GL_POSITION, light_pos)

    light_diffuse = [1.0, 1.0, 1.0, 1.0]
    glLightfv(GL_LIGHT0, GL_DIFFUSE, light_diffuse)
    light_specular = [1.0, 1.0, 1.0, 1.0]
    glLightfv(GL_LIGHT0, GL_SPECULAR, light_specular)

    material_diffuse = [0.8, 0.8, 0.8, 1.0]
    glMaterialfv(GL_FRONT, GL_DIFFUSE, material_diffuse)
    material_specular = [1.0, 1.0, 1.0, 1.0]
    glMaterialfv(GL_FRONT, GL_SPECULAR, material_specular)
    material_shininess = [100.0]
    glMaterialfv(GL_FRONT, GL_SHININESS, material_shininess)


def load_texture(filename):
    img = Image.open(filename)
    img_data = np.array(list(img.getdata()), np.uint8)
    texture_id = glGenTextures(1)
    glBindTexture(GL_TEXTURE_2D, texture_id)
    glTexParameteri(GL_TEXTURE_2D, GL_TEXTURE_MIN_FILTER, GL_LINEAR)
    glTexParameteri(GL_TEXTURE_2D, GL_TEXTURE_MAG_FILTER, GL_LINEAR)
    glTexImage2D(GL_TEXTURE_2D, 0, GL_RGB, img.width, img.height, 0, GL_RGB, GL_UNSIGNED_BYTE, img_data)
    return texture_id


def display(window, texture_id):
    global alpha
    global beta
    global size
    global torus_position
    global torus_velocity
    global frame_count, last_frame_time, fps

    glLoadIdentity()
    glClear(GL_COLOR_BUFFER_BIT)
    glClear(GL_DEPTH_BUFFER_BIT)
    glMatrixMode(GL_PROJECTION)

    def projection():
        alpha_rad = np.radians(alpha)
        beta_rad = np.radians(beta)

        rotate_y = np.array([
            [cos(alpha_rad), 0, sin(alpha_rad), 0],
            [0, 1, 0, 0],
            [-sin(alpha_rad), 0, cos(alpha_rad), 0],
            [0, 0, 0, 1]
        ])

        rotate_x = np.array([
            [1, 0, 0, 0],
            [0, cos(beta_rad), -sin(beta_rad), 0],
            [0, sin(beta_rad), cos(beta_rad), 0],
            [0, 0, 0, 1]
        ])

        glMultMatrixf(rotate_x)
        glMultMatrixf(rotate_y)

    def torus(R, r, N, n):
        glEnable(GL_TEXTURE_2D)
        glBindTexture(GL_TEXTURE_2D, texture_id)

        glCallList(torus_display_list)
        draw_torus_with_vbo()

        glDisable(GL_TEXTURE_2D)

    glLoadIdentity()

    projection()
    R = size
    r = size / 3

    for i in range(3):
        torus_position[i] += torus_velocity[i]

    for i in range(3):
        if torus_position[i] + size > 1.0 or torus_position[i] - size < -1.0:
            torus_velocity[i] *= -1.0

    glTranslatef(torus_position[0], torus_position[1], torus_position[2])
    torus(R, r, 40, 25)

    current_time = time.time()
    delta_time = current_time - last_frame_time
    frame_count += 1
    if delta_time >= 1.0:
        fps = frame_count / delta_time
        print("FPS:", fps)
        frame_count = 0
        last_frame_time = current_time

    glfw.swap_buffers(window)
    glfw.poll_events()


def key_callback(window, key, scancode, action, mods):
    global alpha
    global beta

    if action == glfw.PRESS or action == glfw.REPEAT:
        if key == glfw.KEY_RIGHT:
            alpha += 3
        elif key == glfw.KEY_LEFT:
            alpha -= 3
        elif key == glfw.KEY_UP:
            beta -= 3
        elif key == glfw.KEY_DOWN:
            beta += 3
        elif key == glfw.KEY_F:
            global fill
            fill = not fill
            if fill:
                glPolygonMode(GL_FRONT_AND_BACK, GL_FILL)
            else:
                glPolygonMode(GL_FRONT_AND_BACK, GL_LINE)
        elif key == glfw.KEY_L and action == glfw.PRESS:
            light_enabled = glIsEnabled(GL_LIGHT0)
            if light_enabled:
                glDisable(GL_LIGHT0)
            else:
                glEnable(GL_LIGHT0)


def scroll_callback(window, xoffset, yoffset):
    global size

    if xoffset > 0:
        size -= yoffset / 10
    else:
        size += yoffset / 10


if __name__ == "__main__":
    main()
\end{lstlisting}
\pagebreak
В результате работы программы получилcя следующий вывод:
\begin{center}
    \includegraphics[width=\linewidth]{res1.png}
    \newpage
    \includegraphics[width=\linewidth]{res2.png}
    \newpage
\end{center}
\pagebreak

\begin{table}[h]
\centering
\begin{tabular}{|c|c|}
\hline
\text{Параметр} & \text{Значение (FPS)} \\
\hline
\text{До оптимизации} & 107 \\
\text{После оптимизации} & 200 \\
\hline
\end{tabular}
\caption{Результаты замеров производительности}
\label{tab:performance}
\end{table}

\section{Заключение}
\par
Оптимизация графических приложений на основе OpenGL включает в себя использование различных методик, таких как дисплейные списки, массивы вершин, правильная настройка источников света и выбор эффективных форматов хранения изображений. Эти методы позволяют существенно улучшить производительность, как это видно из увеличения частоты кадров с 107 до 200 FPS. Таким образом, применение данных оптимизаций является важным этапом разработки высокопроизводительных графических приложений.
\end{document}