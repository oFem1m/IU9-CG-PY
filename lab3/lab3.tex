\documentclass[a4paper, 14pt]{extarticle}

% Поля
%--------------------------------------
\usepackage{geometry}
\usepackage[T2A]{fontenc}
\usepackage[russian]{babel}
\usepackage{minted}
\usepackage{float}
\usepackage{graphicx} 
\geometry{a4paper,tmargin=2cm,bmargin=2cm,lmargin=3cm,rmargin=1cm}
%--------------------------------------


%Russian-specific packages
%--------------------------------------
\usepackage[T2A]{fontenc}
\usepackage[utf8]{inputenc} 
\usepackage[english, main=russian]{babel}
%--------------------------------------
\usepackage{textcomp}

% Красная строка
%--------------------------------------
\usepackage{indentfirst}               
%--------------------------------------             


%Graphics
%--------------------------------------
\usepackage{graphicx}
\graphicspath{ {./images/} }
\usepackage{wrapfig}
%--------------------------------------

% Полуторный интервал
%--------------------------------------
\linespread{1.3}                    
%--------------------------------------

%Выравнивание и переносы
%--------------------------------------
% Избавляемся от переполнений
\sloppy
% Запрещаем разрыв страницы после первой строки абзаца
\clubpenalty=10000
% Запрещаем разрыв страницы после последней строки абзаца
\widowpenalty=10000
%--------------------------------------

%Списки
\usepackage{enumitem}

%Подписи
\usepackage{caption} 

%Гиперссылки
\usepackage{hyperref}

\hypersetup {
	unicode=true
}

%Рисунки
%--------------------------------------
\DeclareCaptionLabelSeparator*{emdash}{~--- }
\captionsetup[figure]{labelsep=emdash,font=onehalfspacing,position=bottom}
%--------------------------------------

\usepackage{tempora}
\usepackage{amsmath}
\usepackage{color}
\usepackage{listings}
\lstset{
  belowcaptionskip=1\baselineskip,
  breaklines=true,
  frame=L,
  xleftmargin=\parindent,
  language=Python,
  showstringspaces=false,
  basicstyle=\footnotesize\ttfamily,
  keywordstyle=\bfseries\color{blue},
  commentstyle=\itshape\color{purple},
  identifierstyle=\color{black},
  stringstyle=\color{red},
}

%--------------------------------------
%			НАЧАЛО ДОКУМЕНТА
%--------------------------------------

\begin{document}

%--------------------------------------
%			ТИТУЛЬНЫЙ ЛИСТ
%--------------------------------------
\begin{titlepage}
\thispagestyle{empty}
\newpage


%Шапка титульного листа
%--------------------------------------
\vspace*{-60pt}
\hspace{-65pt}
\begin{minipage}{0.3\textwidth}
\hspace*{-20pt}\centering
\includegraphics[width=\textwidth]{emblem}
\end{minipage}
\begin{minipage}{0.67\textwidth}\small \textbf{
\vspace*{-0.7ex}
\hspace*{-6pt}\centerline{Министерство науки и высшего образования Российской Федерации}
\vspace*{-0.7ex}
\centerline{Федеральное государственное бюджетное образовательное учреждение }
\vspace*{-0.7ex}
\centerline{высшего образования}
\vspace*{-0.7ex}
\centerline{<<Московский государственный технический университет}
\vspace*{-0.7ex}
\centerline{имени Н.Э. Баумана}
\vspace*{-0.7ex}
\centerline{(национальный исследовательский университет)>>}
\vspace*{-0.7ex}
\centerline{(МГТУ им. Н.Э. Баумана)}}
\end{minipage}
%--------------------------------------

%Полосы
%--------------------------------------
\vspace{-25pt}
\hspace{-35pt}\rule{\textwidth}{2.3pt}

\vspace*{-20.3pt}
\hspace{-35pt}\rule{\textwidth}{0.4pt}
%--------------------------------------

\vspace{1.5ex}
\hspace{-35pt} \noindent \small ФАКУЛЬТЕТ\hspace{80pt} <<Информатика и системы управления>>

\vspace*{-16pt}
\hspace{47pt}\rule{0.83\textwidth}{0.4pt}

\vspace{0.5ex}
\hspace{-35pt} \noindent \small КАФЕДРА\hspace{50pt} <<Теоретическая информатика и компьютерные технологии>>

\vspace*{-16pt}
\hspace{30pt}\rule{0.866\textwidth}{0.4pt}
  
\vspace{11em}

\begin{center}
\Large {\bf Лабораторная работа № 3} \\
\large {\bf по курсу <<Компьютерная графика>>} \\
\large <<Модельно-видовые и проективные пеобразования>>
\end{center}\normalsize

\vspace{8em}


\begin{flushright}
  {Студент группы ИУ9-42Б Волохов А. В.\hspace*{15pt} \\
  \vspace{2ex}
  Преподаватель Цалкович П. А.\hspace*{15pt}}
\end{flushright}

\bigskip

\vfill


\begin{center}
\textsl{Москва 2024}
\end{center}
\end{titlepage}
%--------------------------------------
%		КОНЕЦ ТИТУЛЬНОГО ЛИСТА
%--------------------------------------

\renewcommand{\ttdefault}{pcr}

\setlength{\tabcolsep}{3pt}
\newpage
\setcounter{page}{2}

\section{Задача}\label{Sect::task}
\par
Для выполнения лабораторной нужно будет взять код, который вы получили после лабораторной 2, отключить проективные преобразования и заменить куб на фигуру указанную в варианте.
\par
Вариант: круговой тор

\section{Основная теория}

\par
В данном коде реализован пример трехмерной графики с использованием библиотеки OpenGL и библиотеки GLFW для создания окна. Программа отображает тороидальную поверхность (тор) в пространстве.
\par
Подход к инициализации окна и контекста OpenGL аналогичен предыдущему коду. Окно создается с использованием GLFW, а затем устанавливается текущий контекст OpenGL.
\par
Функция display() отвечает за отрисовку сцены. Вначале происходит очистка буферов цвета и глубины. Затем устанавливается матрица проекции с помощью функции projection(). В данном случае происходит вращение сцены вокруг оси X и Y. После этого вызывается функция torus(), которая отрисовывает тор, используя примитивы OpenGL.
\par
Функции key callback() и scroll callback() отвечают за обработку нажатий клавиш и колеса мыши соответственно. При нажатии клавиш стрелок происходит изменение углов поворота сцены. При прокрутке колеса мыши изменяется размер тора.

\pagebreak
\section{Практическая реализация}
Код представлен в Листинге 1.
\par
\begin{center}
    Листинг 1 - lab3.py
\end{center}

\begin{lstlisting}
import math

import glfw
import numpy as np
from OpenGL.GL import *
from math import cos, sin

alpha = 0
beta = 0
size = 0.5
fill = True


def main():
    if not glfw.init():
        return
    window = glfw.create_window(640, 640, "LAB 3", None, None)
    if not window:
        glfw.terminate()
        return
    glfw.make_context_current(window)
    glfw.set_key_callback(window, key_callback)
    glfw.set_scroll_callback(window, scroll_callback)

    glEnable(GL_DEPTH_TEST)
    glDepthFunc(GL_LESS)
    while not glfw.window_should_close(window):
        display(window)
    glfw.destroy_window(window)
    glfw.terminate()


def display(window):
    glLoadIdentity()
    glClear(GL_COLOR_BUFFER_BIT)
    glClear(GL_DEPTH_BUFFER_BIT)
    glMatrixMode(GL_PROJECTION)
    global alpha
    global beta

    def projection():
        alpha_rad = np.radians(alpha)
        beta_rad = np.radians(beta)

        rotate_y = np.array([
            [cos(alpha_rad), 0, sin(alpha_rad), 0],
            [0, 1, 0, 0],
            [-sin(alpha_rad), 0, cos(alpha_rad), 0],
            [0, 0, 0, 1]
        ])

        rotate_x = np.array([
            [1, 0, 0, 0],
            [0, cos(beta_rad), -sin(beta_rad), 0],
            [0, sin(beta_rad), cos(beta_rad), 0],
            [0, 0, 0, 1]
        ])

        glMultMatrixf(rotate_x)
        glMultMatrixf(rotate_y)

    def torus(R, r, N, n):
        for i in range(N):
            for j in range(n):
                theta = (2 * math.pi / N) * i
                phi = (2 * math.pi / n) * j
                theta_next = (2 * math.pi / N) * (i + 1)
                phi_next = (2 * math.pi / n) * (j + 1)

                # Рисуем "квадраты"

                x0 = (R + r * cos(phi)) * cos(theta)
                y0 = (R + r * cos(phi)) * sin(theta)
                z0 = r * sin(phi)

                x1 = (R + r * cos(phi)) * cos(theta_next)
                y1 = (R + r * cos(phi)) * sin(theta_next)
                z1 = r * sin(phi)

                x2 = (R + r * cos(phi_next)) * cos(theta_next)
                y2 = (R + r * cos(phi_next)) * sin(theta_next)
                z2 = r * sin(phi_next)

                x3 = (R + r * cos(phi_next)) * cos(theta)
                y3 = (R + r * cos(phi_next)) * sin(theta)
                z3 = r * sin(phi_next)

                glBegin(GL_QUADS)
                glVertex3f(x0, y0, z0)
                glVertex3f(x1, y1, z1)
                glVertex3f(x2, y2, z2)
                glVertex3f(x3, y3, z3)
                glEnd()

    glLoadIdentity()

    projection()
    R = size
    r = size / 3

    glColor3f(1.0, 0.0, 0.0)

    torus(R, r, 40, 25)

    glfw.swap_buffers(window)
    glfw.poll_events()


def key_callback(window, key, scancode, action, mods):
    global alpha
    global beta
    if action == glfw.PRESS or action == glfw.REPEAT:
        if key == glfw.KEY_RIGHT:
            alpha += 3
        elif key == glfw.KEY_LEFT:
            alpha -= 3
        elif key == glfw.KEY_UP:
            beta -= 3
        elif key == glfw.KEY_DOWN:
            beta += 3
        elif key == glfw.KEY_F:
            global fill
            fill = not fill
            if fill:
                glPolygonMode(GL_FRONT_AND_BACK, GL_FILL)
            else:
                glPolygonMode(GL_FRONT_AND_BACK, GL_LINE)


def scroll_callback(window, xoffset, yoffset):
    global size

    if xoffset > 0:
        size -= yoffset / 10
    else:
        size += yoffset / 10


if __name__ == "__main__":
    main()

\end{lstlisting}
В результате работы программы получилcя следующий вывод:
\begin{center}
    \includegraphics[width=\linewidth]{res1.png}
    \newpage
    \includegraphics[width=\linewidth]{res2.png}
\end{center}
\pagebreak

\section{Заключение}
В результате выполнения лабораторной работы был создан пример трехмерной графики с помощью библиотеки OpenGL. Реализована отрисовка тороидальной поверхности, а также добавлена возможность вращения и изменения размера объекта с помощью клавиш и колеса мыши. Этот пример демонстрирует основные принципы работы с трехмерной графикой в контексте OpenGL и может быть использован в дальнейшем для изучения более сложных трехмерных моделей и алгоритмов отображения.
\end{document}